%%%%%%%%%%%%%%%%%%%%%%%%%%%%%%%%%%%%%%%%%%%
%%% DOCUMENT PREAMBLE %%%
\documentclass[12pt]{report}
\usepackage[english]{babel}
%\usepackage{natbib}
\usepackage{url}
\usepackage[utf8x]{inputenc}
\usepackage{amsmath}
\usepackage{graphicx}
\graphicspath{{images/}}
\usepackage{parskip}
\usepackage{fancyhdr}
\usepackage{vmargin}
\setmarginsrb{3 cm}{2.5 cm}{3 cm}{2.5 cm}{1 cm}{1.5 cm}{1 cm}{1.5 cm}

\title{Tutorial for Nautilus}								
% Title
\author{ Sacha Gavino}						
% Author
\date{today's date}
% Date

\makeatletter
\let\thetitle\@title
\let\theauthor\@author
\let\thedate\@date
\makeatother

\pagestyle{fancy}
\fancyhf{}
\rhead{\theauthor}
\lhead{\thetitle}
\cfoot{\thepage}
%%%%%%%%%%%%%%%%%%%%%%%%%%%%%%%%%%%%%%%%%%%%
\begin{document}

%%%%%%%%%%%%%%%%%%%%%%%%%%%%%%%%%%%%%%%%%%%%%%%%%%%%%%%%%%%%%%%%%%%%%%%%%%%%%%%%%%%%%%%%%

\begin{titlepage}
	\centering
    \vspace*{0.5 cm}
   % \includegraphics[scale = 0.075]{bsulogo.png}\\[1.0 cm]	% University Logo


	\rule{\linewidth}{0.2 mm} \\[0.4 cm]
	{ \huge \bfseries \thetitle}\\
	\rule{\linewidth}{0.2 mm} \\[1.5 cm]
	
	\begin{minipage}{0.4\textwidth}
		\begin{flushleft} \large
		%	\emph{Submitted To:}\\
		%	Name\\
          % Affiliation\\
           %contact info\\
			\end{flushleft}
			\end{minipage}~
			\begin{minipage}{0.4\textwidth}
            
			\begin{flushright} \large

			Sacha Gavino 

		\end{flushright}
           
	\end{minipage}\\[2 cm]
	
	\includegraphics[scale = 0.5]{../nautilus_logo.pdf}
    
    
    
    
	
\end{titlepage}

%%%%%%%%%%%%%%%%%%%%%%%%%%%%%%%%%%%%%%%%%%%%%%%%%%%%%%%%%%%%%%%%%%%%%%%%%%%%%%%%%%%%%%%%%

\tableofcontents
\pagebreak

%%%%%%%%%%%%%%%%%%%%%%%%%%%%%%%%%%%%%%%%%%%%%%%%%%%%%%%%%%%%%%%%%%%%%%%%%%%%%%%%%%%%%%%%%
\renewcommand{\thesection}{\arabic{section}}
\section{Section 1}
 

The paper will look professional and is typed in LaTeX. Make sure the cover page       includes the Project Title, Team Names, Date, and PIC Math Logo (1 point).

Remaining pages should include: (7 points) 

1.Problem summary: state the industrial problem, its context, and its significance; include a list of most important questions to be addressed. State the industrial sponsor of the project. 

2.Statement of your results, with assumptions and questions(one page).

3.Details about your (one of several) approaches to problem; elaborate which questions are important and why; your results and/or anticipations (this is where your mathematics details fit); relevant information from your sources (citing for example \cite{ACAMP}) and what the team is incorporating from each source; details of your solution, if any; relevant tables and graphs (tables and graphs pages do not count towards the page limit). State your work and include references - you might never make use of reference \cite{ACMT} but it is a great paper.

4.Conclusions with questions, implications and ideas for future work

5.Extended Bibliography List (not included in page limit) in the following format: 

[1] Author name(s). Paper title. Journal name, volume XY, pages ab-cd, year. 

6.Acknowledgments page, stating:

?PIC Math is a program of the Mathematical Association of America (MAA) and the Society for Industrial and AppliedMathematics (SIAM). Support is provided by the National Science Foundation (NSF grant DMS-1345499).?

7.Advisor names, etc. 

Flow /Visual Composition/Grammar/Spelling (2 points) 

\newpage
\section{Acknowledgments}

 PIC Math is a program of the Mathematical Association of America (MAA) and the Society for Industrial and Applied Mathematics (SIAM). Support is provided by the National Science Foundation (NSF grant DMS-1345499).
\newpage
 
\begin{thebibliography}{111}
   
  \bibitem{ACMT}
A. Aldroubi, C. Cabrelli, U. Molter, and Sui Tang,
Dynamical sampling, 
{\it  Applied and Computational Harmonic Analysis}, doi:10.1016/j.acha.2015.08.014, 2016

%if the "underfill \hbox" warning bothers you uncomment the following line
%\raggedright
\bibitem{ACAMP}
    A. Aldroubi, C. Cabrelli, A. F. Cakmak, U. Molter,  and A. Petrosyan,
    Iterative actions of normal operators, 
    Submitted. Available at http://arxiv.org/abs/1602.04527.
  
\bibitem{Gro01} 
    K. Groechenig,
    {\it Foundations of time-frequency analysis}, 
    Birkh\"auser Boston, 2001.

\end{thebibliography}
\end{document}

%This template was created by Roza Aceska.